%%%%%%%%%%%%%%%%%%%%%%%%%%%%%%%%%%%%%%%%%%%%%%%%%%%%%%%%%%%%%%%%%%%%%%
% LaTeX Example: Project Report
%
% Source: http://www.howtotex.com
%
% Feel free to distribute this example, but please keep the referral
% to howtotex.com
% Date: March 2011 
% 
%%%%%%%%%%%%%%%%%%%%%%%%%%%%%%%%%%%%%%%%%%%%%%%%%%%%%%%%%%%%%%%%%%%%%%
% How to use writeLaTeX: 
%
% You edit the source code here on the left, and the preview on the
% right shows you the result within a few seconds.
%
% Bookmark this page and share the URL with your co-authors. They can
% edit at the same time!
%
% You can upload figures, bibliographies, custom classes and
% styles using the files menu.
%
% If you're new to LaTeX, the wikibook is a great place to start:
% http://en.wikibooks.org/wiki/LaTeX
%
%%%%%%%%%%%%%%%%%%%%%%%%%%%%%%%%%%%%%%%%%%%%%%%%%%%%%%%%%%%%%%%%%%%%%%
% Edit the title below to update the display in My Documents
%\title{Project Report}
%



%%% Preamble
\documentclass[paper=a4, fontsize=11pt]{scrartcl}
\usepackage{fourier}
\usepackage[table,xcdraw]{xcolor}
\usepackage[protrusion=true,expansion=true]{microtype}	
\usepackage{amsmath,amsfonts,amsthm} % Math packages
\usepackage{graphicx}	
\usepackage{url}
\usepackage{bookmark}

%%%Greek and english
\usepackage{polyglossia}
\setdefaultlanguage{greek}
\setotherlanguages{english}
%%% Bibliography 
\usepackage{biblatex}
\addbibresource{ref.bib}
 
%%%center tables
\usepackage{float}
\restylefloat{table}
%%%% algorithms
\usepackage{algorithm} 
\usepackage{algpseudocode} 

%%%%%images
\graphicspath{ {./images/} }


%%%fonts
\usepackage{fontspec}
\setmainfont{FreeSerif}[
  Extension      = .otf,
  UprightFont    = *,
  ItalicFont     = *Italic,
  BoldFont       = *Bold,
  BoldItalicFont = *BoldItalic
]
\setsansfont{FreeSans}[
  Extension      = .otf,
  UprightFont    = *,
  ItalicFont     = *Oblique,
  BoldFont       = *Bold,
  BoldItalicFont = *BoldOblique,
]
\setmonofont{FreeMono}[
  Extension      = .otf,
  UprightFont    = *,
  ItalicFont     = *Oblique,
  BoldFont       = *Bold,
  BoldItalicFont = *BoldOblique,
]



%%% Custom sectioning
\usepackage{sectsty}
\allsectionsfont{\centering \normalfont\scshape}


%%% Custom headers/footers (fancyhdr package)
\usepackage{fancyhdr}
\pagestyle{fancyplain}
\fancyhead{}											% No page header
\fancyfoot[L]{}											% Empty 
\fancyfoot[C]{}											% Empty
\fancyfoot[R]{\thepage}									% Pagenumbering
\renewcommand{\headrulewidth}{0pt}			% Remove header underlines
\renewcommand{\footrulewidth}{0pt}				% Remove footer underlines
\setlength{\headheight}{13.6pt}


%%% Equation and float numbering
\numberwithin{equation}{section}		% Equationnumbering: section.eq#
\numberwithin{figure}{section}			% Figurenumbering: section.fig#
\numberwithin{table}{section}				% Tablenumbering: section.tab#

%%%paragraph 
\setlength{\parindent}{1em}
\setlength{\parskip}{0.5em}

\usepackage{titlesec, lipsum}
\titleformat{\paragraph}[block]{\filcenter}{}{0pt}{}

%%% Maketitle metadata
\newcommand{\horrule}[1]{\rule{\linewidth}{#1}} 	% Horizontal rule



\title{
		%\vspace{-1in} 	
		\usefont{OT1}{bch}{b}{n}
		\normalfont \normalsize \textsc{Πανεπιστήμιο Ιωαννίνων Τμήμα Πληροφορικής και Τηλεπικοινωνίων} \\ [1em]
		\normalfont \normalsize \textsc{Αλγόριθμοι και πολυπλοκότητα} \\ [1em]
		\horrule{0.5pt} \\[0.4cm]
		\huge To πρόβλημα χρονοπρογραμματισμού εργασιών σε βιομηχανικό περιβάλλον. \\
		\horrule{2pt} \\[0.5cm]
}
\author{
		\normalfont 								\normalsize
        Αριστείδης Δραγάτης\\[-1pt]		\normalsize
	    2111 \\[-1pt]   \normalsize
        \today
}
\date{}

%%% Begin document
\begin{document}

\maketitle
%%% Abstract

\section*{Ερώτημα 1}
Job Shop Scheduling: \\
Στο Job Shop Scheduling έχουμε n εργασίες διαφορετικών χρόνων επεξεργασίας, οι οποίες πρέπει να προγραμματιστούν σε μηχανές m με διαφορετική επεξεργαστική ισχύ, ενώ προσπαθούμε να ελαχιστοποιήσουμε το makespan ( τη συνολική διάρκεια του χρονοδιαγράμματος).
i)Κάθε εργασία (job) έχει το δικό της ξεχωριστό μονοπάτι. 

Permutation Flow Shop: \\ 
Είναι ένας ειδικός τύπος προβλήματος προγραμματισμού flow shop, στο οποίο η σειρά επεξεργασίας των εργασιών στους πόρους είναι η ίδια για κάθε επόμενο βήμα επεξεργασίας.
i)Όλες οι εργασίες (jobs) έχουν την ίδια σειρά επεξεργασίας μέσα στις μηχανές. \\
και ii) κάθε μηχανή επεξεργάζεται τις εργασίες με την ίδια σειρά.

Flow Shop:\\
*Στο Flow Shop πρόβλημα m μηχανές που πρέπει να επεξεργαστούν n εργασίες ( jobs ), 
όπου υπάρχει αυστηρή σειρά όλων των λειτουργιών που πρέπει να εκτελεστούν σε όλες τις εργασίες.
Η πρώτη συνθήκη είναι ίδια με το permutation flow shop δηλαδή: \\
i)Όλες οι εργασίες (jobs ) έχουν την ίδια σειρά επεξεργασίας μέσα στις μηχανές. \\
Ενώ η δεύτερη συνθήκη αλλάζει: \\
ii) Η σειρά των εργασιών σε κάθε μηχανή μπορεί να είναι διαφορετικός. ( πχ. σαν να υπάρχει ένα buffer ανάμεσα στις εργασίες και μπορούμε να αλλάξουμε την σειρά των εργασιών σε κάθε μηχανή ).


Open Shop: \\
Σε αυτή την περίπτωση κάθε εργασία αποτελείται από ένα σύνολο λειτουργιών που πρέπει να υποβληθούν σε επεξεργασία με αυθαίρετη σειρά, δηλαδή η σειρά με την οποία συμβαίνουν τα βήματα επεξεργασίας μπορεί να ποικίλει ελεύθερα. \\ 

NP-Hard meaning:
Είναι η καθοριστική ιδιότητα μιας κατηγορίας προβλημάτων που είναι ανεπίσημα "τουλάχιστον τόσο σκληρά όσο τα δυσκολότερα προβλήματα στο NP". \\ 
Δηλαδή ένα πρόβλημα απόφασης H είναι NP-hard όταν για κάθε πρόβλημα L στο NP, υπάρχει μια πολυωνυμική μείωση πολλά προς ένα από L σε H.  	

	
\section*{Ερώτημα 2}
Η δομή δεδομένων που επέλεξα είναι ένα λεξικό, όπου κλειδί είναι το όνομα κάθε στιγμιότυπου προβλήματος, και τιμή είναι τα περιεχόμενα κάθε στιγμιότυπου προβλήματος. Πιο συγκεκριμένα χρησιμοποιήθηκε η εντολή with open() μέσα σε μια for loop η οποία διαβάζει κάθε αρχείο ξεχωριστά και στη συνέχεια τα περνάμε στη δομή dictionary. 
    
    
\section*{Ερώτημα 3}
Ο dispatching rule που επέλεξα για την επίλυση των στιγμιοτύπων προβλημάτων JSSP, είναι το Shortest processing time (SPT). \\
Solving JSSP instance in file txt\_files/la01.txt: \\
Number of jobs: 10 \\
Number of machines: 5 \\
Job processing order (SPT rule): [3, 2, 9, 5, 1, 2, 8, 9, 2, 3, 1, 5, 5, 8, 3, 8, 3, 6, 10, 9, 9, 2, 1, 6, 1, 4, 8, 6, 5, 4, 7, 2, 10, 4, 4, 7, 10, 4, 6, 10, 5, 8, 7, 7, 6, 7, 1, 10, 3, 9] \\


Solving JSSP instance in file txt\_files/la02.txt: \\
Number of jobs: 10 \\
Number of machines: 5  \\
Job processing order (SPT rule): [7, 9, 1, 5, 2, 10, 7, 1, 3, 2, 2, 5, 6, 3, 7, 1, 2, 10, 9, 5, 4, 5, 5, 6, 7, 8, 9, 9, 3, 10, 6, 8, 6, 3, 10, 9, 1, 4, 10, 7, 8, 2, 4, 1, 4, 8, 4, 6, 8, 3] \\


Solving JSSP instance in file txt\_files/la03.txt: \\
Number of jobs: 10 \\
Number of machines: 5 \\
Job processing order (SPT rule): [9, 10, 8, 10, 6, 3, 2, 10, 7, 7, 2, 1, 8, 2, 5, 8, 7, 4, 1, 3, 9, 10, 2, 1, 2, 3, 3, 3, 4, 9, 8, 9, 4, 5, 5, 4, 9, 7, 5, 4, 6, 5, 6, 1, 10, 1, 8, 6, 6, 7] \\


Solving JSSP instance in file txt\_files/la04.txt: \\
Number of jobs: 10 \\ 
Number of machines: 5 \\ 
Job processing order (SPT rule): [9, 10, 8, 10, 6, 3, 2, 10, 7, 7, 2, 1, 8, 2, 5, 8, 7, 4, 1, 3, 9, 10, 2, 1, 2, 3, 3, 3, 4, 9, 8, 9, 4, 5, 5, 4, 9, 7, 5, 4, 6, 5, 6, 1, 10, 1, 8, 6, 6, 7] \\


Solving JSSP instance in file txt\_files/la05.txt: \\
Number of jobs: 10 \\
Number of machines: 5 \\
Job processing order (SPT rule): [2, 6, 9, 7, 8, 10, 4, 3, 3, 5, 10, 5, 5, 10, 5, 6, 7, 8, 4, 2, 4, 7, 8, 2, 9, 6, 3, 4, 2, 9, 9, 7, 2, 3, 8, 4, 1, 10, 1, 7, 1, 5, 6, 6, 9, 1, 8, 10, 1, 3] \\


Solving JSSP instance in file txt\_files/mt06.txt: \\
Number of jobs: 6 \\ 
Number of machines: 6 \\
Job processing order (SPT rule): [1, 3, 5, 6, 1, 1, 4, 5, 5, 6, 6, 2, 3, 5, 6, 2, 3, 4, 4, 4, 5, 1, 1, 1, 3, 2, 3, 4, 3, 4, 5, 6, 2, 2, 2, 6] \\


Solving JSSP instance in file txt\_files/mt10.txt: \\
Number of jobs: 10 \\ 
Number of machines: 10 \\
Job processing order (SPT rule): [10, 5, 10, 10, 10, 10, 4, 5, 10, 9, 10, 10, 3, 10, 2, 10, 1, 8, 2, 6, 9, 4, 7, 4, 6, 3, 4, 5, 4, 8, 1, 1, 6, 7, 6, 6, 7, 2, 7, 6, 2, 8, 1, 3, 2, 9, 1, 1, 6, 7, 5, 9, 5, 7, 4, 8, 3, 5, 9, 4, 6, 4, 6, 9, 9, 5, 1, 4, 8, 3, 1, 4, 5, 2, 7, 8, 1, 8, 8, 9, 7, 3, 5, 2, 3, 8, 9, 7, 7, 2, 6, 8, 1, 2, 9, 2, 3, 5, 3, 3] \\


Solving JSSP instance in file txt\_files/mt20.txt: \\
Number of jobs: 20 \\
Number of machines: 5 \\
Job processing order (SPT rule): [20, 15, 20, 20, 20, 20, 5, 14, 19, 3, 12, 10, 11, 18, 2, 16, 19, 4, 7, 4, 10, 16, 3, 4, 15, 4, 8, 11, 6, 11, 7, 6, 6, 7, 12, 7, 10, 16, 2, 8, 1, 13, 2, 19, 1, 6, 7, 11, 5, 9, 5, 17, 14, 18, 5, 13, 19, 14, 16, 10, 6, 9, 14, 9, 15, 1, 14, 18, 3, 1, 4, 15, 12, 17, 18, 1, 8, 8, 19, 10, 17, 3, 5, 3, 8, 9, 12, 17, 17, 12, 16, 18, 2, 9, 11, 2, 13, 15, 13, 13]



\section*{Ερώτημα 4} 
.\\
\begin{figure}[htbp]
\centerline{\includegraphics{Job\_Shop\_Scheduling\_Gantt\_chart.png}}
\caption{Screenshot από το αποτέλεσμα του γραφήματος Gantt Chart με την λύση του Job Shop Scheduling Problem με τη χρήση του (Shortest processing time) dispatching rule για το αρχείο la01.txt .}
\label{fig}
\end{figure}


\section{Ερώτημα 5}


\section{Συμπεράσματα}


% \printbibliography
\end{document}